\documentclass{article}
\usepackage{goststyle} % Подключение созданного стиля

\usepackage{lipsum} % Для генерации "рыбы" (текст-заполнитель)


\begin{document}

 \begin{center} ГУАП\\ \vspace{1cm} КАФЕДРА №
\_\_ \\ \vspace{3cm} 
 \end{center}
ОТЧЕТ\\  ЗАЩИЩЕН С ОЦЕНКОЙ
\vspace{0.2cm}\\
ПРЕПОДАВАТЕЛЬ \vspace{0.2cm}\\ 
\begin{tabularx}{\textwidth}{|>{\centering\arraybackslash}p{10cm}|>{\centering\arraybackslash}p{3cm}|>{\centering\arraybackslash}p{4cm}|}  
& & \\ \hline 
    \small{должность, уч. степень, звание} &
\small{подпись, дата} & \small{инициалы, фамилия}   \end{tabularx}

\vspace{3cm} \textbf{ОТЧЕТ О ЛАБОРАТОРНОЙ РАБОТЕ} \\ \vspace{1cm} по курсу:
\underline{\hspace{6cm}} \\ \vspace{1cm}

\vfill \begin{flushright} РАБОТУ ВЫПОЛНИЛ \\ \vspace{0.5cm}
\begin{tabularx}{\textwidth}{|X|X|X|} \hline СТУДЕНТ гр. № & подпись, дата &
инициалы, фамилия \\ \hline & & \\ \hline \end{tabularx} \end{flushright}

\vspace{2cm} 
\begin{center} Санкт-Петербург 20\_\_ \end{center} 
\tableofcontents

\structheading{Введение}
\lipsum[1]

\section{Проверка заголовков}
\subsection{Подсекция}
\lipsum[1]
\subsubsection{\lipsum[3]} 
\subsubsection{}\lipsum[4] % Можно и так
\subsubsection{Так тоже можно}\lipsum[4] % Вот так с заголовком, но не по госту 
\subsection{Еще подсекция}
\lipsum[5]
\section{Проверка перечислений}
\begin{itemize}
    \item \lipsum[6]
    \item \lipsum[7]
    \item \lipsum[8]
\end{itemize}
\begin{enumerate}
    \item \lipsum[9]
    \item Вложенны список
    \begin{enumerate}
        \item \lipsum[10]
        \item \lipsum[11]
    \end{enumerate}
        \item \lipsum[12]
\end{enumerate}

\section{Формулы, таблицы и рисунки}
Математическая модель представлена следующим уравнением: \begin{equation} E =
mc^2 \end{equation}

Результаты также можно представить в виде графиков (см. Рисунок \ref{fig:example}).


\begin{figure}[H] \centering 
    \includegraphics[width=0.5\textwidth]
    {example-image} 
    \caption{Пример графика} 
    \label{fig:example}
\end{figure}

\begin{table}[H] \caption{Пример таблицы}\centering \begin{tabular}{|c|c|c|} \hline Столбец 1 & Столбец
2 & Столбец 3 \\ \hline Ячейка 1 & Ячейка 2 & Ячейка 3 \\ Ячейка 4 & Ячейка 5 &
Ячейка 6 \\ \hline \end{tabular}  \end{table}


\structheading{ЗАКЛЮЧЕНИЕ}
Краткое резюме проведенной работы и выводы.
\lipsum[13]

\application{Заголовок приложения}
\lipsum[14]
\begin{figure}[H] \centering \includegraphics[width=0.5\textwidth]
{example-image} \caption{Пример графика} \label{fig:example} \end{figure}

\begin{table}[h] \centering \begin{tabular}{|c|c|c|} \hline Столбец 1 & Столбец
2 & Столбец 3 \\ \hline Ячейка 1 & Ячейка 2 & Ячейка 3 \\ Ячейка 4 & Ячейка 5 &
Ячейка 6 \\ \hline \end{tabular} \caption{Пример таблицы} \end{table}
\end{document}